\documentclass[a4paper, 12pt]{report}

\usepackage[spanish]{babel}
\usepackage[utf8]{inputenc}
\usepackage[left=4cm, right=4cm, top=4cm]{geometry}
\usepackage{textcomp}
\usepackage{booktabs}
\usepackage{amssymb}
\usepackage{bussproofs}
\usepackage{fancyhdr}
\usepackage{graphicx}
\usepackage{amsmath}
\usepackage{enumitem}
\usepackage{ifsym}
\usepackage{float}


\usepackage{hyperref}
\hypersetup{
    colorlinks=true,
    linkcolor=blue,
    filecolor=magenta,
    urlcolor=cyan,
}

\pagestyle{fancy}
\lhead{Almeida, Figueroa \& Ibarra}
\chead{Tarea 5}
\rhead{\today}

\begin{document}
\begin{titlepage}
    \centering
    {\scshape\Huge Universidad Nacional Autónoma de México \par}
    \vspace{1.25cm}
    {\scshape\huge Fundamentos de Bases de Datos\par}
    \vspace{1.25cm}
    {\huge\bfseries Tarea 5:\\ Dependencias y Normalización\par}
    \vspace{1.25cm}
    {\Large\textsc Almeida Rodríguez Jerónimo\par}
    \vspace{.1cm}
    {\large\texttt{418003815}\par}
    \vspace{0.25cm}
    {\Large\textsc Figueroa Sandoval Gerardo Emiliano\par}
    \vspace{.1cm}
    {\large\texttt{315241774}\par}
    \vspace{0.25cm}
    {\Large\textsc Ibarra Moreno Gisselle \par}
    \vspace{.1cm}
    {\large\texttt{315602193}\par}
    \vspace{1.5cm}
    \vfill
    \begin{figure}[hb!]
        \includegraphics[width=.3\textwidth]
            {../logos/escudo_f-ciencias.png}\hfill
        \includegraphics[width=.3\textwidth]
            {../logos/Escudo_UNAM.png}\hfill
    \end{figure}
\end{titlepage}

\begin{enumerate}
\item{
    \textbf{Preguntas de repaso:}
    \begin{itemize}
    \item{¿Qué es una dependencia funcional y cómo se define?\\
        Una dependencia funcional es una relación unidireccional entre dos
        atributos tal que para algún valor de B, solamente tiene relacionado uno
        de A por medio de la relación.\\
        Se definen cómo $A\rightarrow B$.
    }
    \item{¿Para qué sirve el concepto de dependencia en la normalización?\\
        Para eliminar o en su defecto reducir la redundancia en un base de datos.
    }
            funcionales que implica A.\\
    \item{Sea A la llave de R(A, B, C). Indica todas las dependencias
        $$A\rightarrow B;\ \ A\rightarrow C$$
        Y, cómo relación trivial, $A\rightarrow A$.
    }
    \item{¿Qué es una forma normal? ¿Cuál es el objetivo de normalizar un modelo
            de datos?\\
        Una forma normal es una manera de, por medio de reglas sobre las
        relaciones, descomponer los datos de la base de tal manera que se haya
        reducido al redundancia.\\
        El objetivo de normalizr es un modelo de datos es reducir en la mayor
        medida posible la redundancia en una base de datos.
    }
    \item{¿En qué casos es preferible lograr 3NF en vez de BCNF?\\
        Cuándo no se desea descomponer más el esquema.
    }
    \end{itemize}
}
\item {Proporciona algunos ejemplos que demuestren que las siguientes reglas
	no son válidas:
\begin{enumerate}
	\item Si A → B, entonces B → A\\
	Sea la relación R(restaurante,ciudad,teléfono).\\
	Se tiene la DF $restaurante \rightarrow ciudad$, ya que un restaurante tiene
	asociada una sola ciudad, pero notemos que la DF $ciudad \rightarrow
	restaurante$ no se da, ya que una ciudad tiene varios restaurantes. \\
	Por lo que A $\rightarrow$ B no implica que B $\rightarrow$ A
\end{enumerate}}
\item {Para cada uno de los esquemas que se muestran a continuación:
\begin{enumerate}
	\item R(A,B,C,D,E) con F = \{AB $\rightarrow$ CD, E $\rightarrow$ C, D
		$\rightarrow$ B\}
	\item R(A,B,C,D,E)con F = \{AB $\rightarrow$ C, DE $\rightarrow$ C, B $\rightarrow$ D\}
\end{enumerate}
\begin{itemize}
	\item Especifica de  ser  posible dos DF  no  triviales que  se  pueden
	derivar  de  las  dependencias funcionales dadas.\\
	a. \{AB $\rightarrow$ C, AB $\rightarrow$ D\}.\\
	\item Indica alguna llave candidata para R.\\
	a.\{ABE\}+=\{ABCDE\}\\
	\item Especifica todas las violaciones a la BCNF\\
	a.\{AB $\rightarrow$ CD, E $\rightarrow$ C, D
	$\rightarrow$ B\}
	\item Normaliza de acuerdo a BCNF, asegúrate de indicar cuáles son las
	relaciones resultantes con sus respectivas dependencias funcionales:\\
	a. Tomamos la violación AB $\rightarrow$ CD.\\
	Obtenemos las relación S(A,B,C,D) con dependecias \{AB $\rightarrow$ CD,
	D $\rightarrow$ B\} y la relación T(A,B,E) con dependencias \{ABE $\rightarrow$ ABE\}.\\
	\{AB\}+=\{ABCD\} es una llave para S, entonces tomamos la violación D $\rightarrow$ B.\\
	Obtenemos la relación U(D,B) con dependencia \{D $\rightarrow$ B\} y la
	relación V(D,A,C) con dependencia \{DAC $\rightarrow$ DAC\}.\\
	Por lo tanto el esquema en BCNF es U(D,B), V(D,A,C) y T(A,B,E).
\end{itemize}}
\item Para cada una de  las  siguientes  relaciones  con  su  respectivo  conjunto  de  dependencias funcionales:
\begin{enumerate}
	\item R(A,B,C,D,E,F)con F = \{B $\rightarrow$ D, B $\rightarrow$ E, D $\rightarrow$ F, AB $\rightarrow$ C\}
	\item R(A,B,C,D,E)con F = \{A $\rightarrow$ BC, B $\rightarrow$ D, CD$\rightarrow$ E, E $\rightarrow$ A \}
\end{enumerate}
\begin{itemize}
	\item Indica todas las violaciones a la 3NF\\
	b.\{A\}+ = \{ABCDE\} y \{E\}+ = \{EABCD\} son llaves, entonces la única
	violación a la 3NF es B $\rightarrow$ D.
	\item Normaliza de acuerdo a la 3NF\\
	b. Superfluos por la izquiera:  CD $\rightarrow$ E\\
	¿C es superfluo? D $\rightarrow$ E, \{D\}+=\{D\} entonces C no es superfluo.\\
	¿D es superfluo? C $\rightarrow$ E, \{C\}+=\{C\} entonces D no es superfluo.\\
	Superfluos por la derecha: A $\rightarrow$ BC\\
	¿B es superfluo? A $\rightarrow$ C, F'=\{A$\rightarrow$C, B$\rightarrow$D, CD$\rightarrow$E, E$\rightarrow$A\}.\\
	\{A\}+= \{AC\} por lo tanto, B no es superfluo.\\
	¿C es superfluo? A$\rightarrow$B F''=\{A$\rightarrow$B, B$\rightarrow$D, CD$\rightarrow$E, E$\rightarrow$A\}.\\
	\{A\}+= \{AB\} por lo tanto, C no es superfluo.\\
	Entonces F ya es el mínimo.\\
	S(A,B,C), T(B,D), U(C,D,E) y V(E,A) es el esquema en 3NF.
\end{itemize}
\item

\item Para  cada  uno  de  los  esquemas,  con  su  respectivo  conjunto  de  dependencias  multivaluadas, resuelve los siguientes puntos:\\
\begin{enumerate}
	\item R(A,B,C,D)con DMV = {AB $\twoheadrightarrow$C, B $\rightarrow$ D}
	\item R(A,B,C,D,E)con DMV = { A $\twoheadrightarrow$B, AB $\rightarrow$ C, A $\rightarrow$ D, AB $\rightarrow$ E}\\
\end{enumerate}
\begin{itemize}
	\item Encuentra todas las violaciones a la 4NF.
	\{AB\}+=\{ABCDE\} es una llave. Las violaciones son \{A$\twoheadrightarrow$B\}.\\
	\item Normaliza de acuerdo a la 4NF.\\
	Tomamos la violación A$\twoheadrightarrow$B.\\
	Obtenemos las relaciones S(A,B) con DMV=\{A$\twoheadrightarrow$B\} y
	T(A,C,D,E) con DMV=\{A$\rightarrow$\}.\\
	\{ACE\}+=\{ACED\} es una llave para T, entonces tomamos la violación
	A$\rightarrow$D.\\
	Obtenemos las relaciones U(A,D) con DMV=\{A$\rightarrow$\} y V(A,C,E) con
	DMV=\{ACE$\rightarrow$\}.\\
	Por lo tanto, el esquema en 4NF es S(A,B), U(A,D) y V(A,C,E).
\end{itemize}
\end{enumerate}

\end{document}
