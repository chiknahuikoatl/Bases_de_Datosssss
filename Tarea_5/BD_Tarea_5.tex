\documentclass[a4paper, 12pt]{report}

\usepackage[spanish]{babel}
\usepackage[utf8]{inputenc}
\usepackage[left=4cm, right=4cm, top=4cm]{geometry}
\usepackage{textcomp}
\usepackage{booktabs}
\usepackage{amssymb}
\usepackage{bussproofs}
\usepackage{fancyhdr}
\usepackage{graphicx}
\usepackage{amsmath}
\usepackage{enumitem}
\usepackage{ifsym}
\usepackage{float}


\usepackage{hyperref}
\hypersetup{
    colorlinks=true,
    linkcolor=blue,
    filecolor=magenta,
    urlcolor=cyan,
}

\pagestyle{fancy}
\lhead{Almeida, Figueroa \& Ibarra}
\chead{Tarea 5}
\rhead{\today}

\begin{document}
\begin{titlepage}
    \centering
    {\scshape\Huge Universidad Nacional Autónoma de México \par}
    \vspace{1.25cm}
    {\scshape\huge Fundamentos de Bases de Datos\par}
    \vspace{1.25cm}
    {\huge\bfseries Tarea 5:\\ Dependencias y Normalización\par}
    \vspace{1.25cm}
    {\Large\textsc Almeida Rodríguez Jerónimo\par}
    \vspace{.1cm}
    {\large\texttt{418003815}\par}
    \vspace{0.25cm}
    {\Large\textsc Figueroa Sandoval Gerardo Emiliano\par}
    \vspace{.1cm}
    {\large\texttt{315241774}\par}
    \vspace{0.25cm}
    {\Large\textsc Ibarra Moreno Gisselle \par}
    \vspace{.1cm}
    {\large\texttt{315602193}\par}
    \vspace{1.5cm}
    \vfill
    \begin{figure}[hb!]
        \includegraphics[width=.3\textwidth]
            {../logos/escudo_f-ciencias.png}\hfill
        \includegraphics[width=.3\textwidth]
            {../logos/Escudo_UNAM.png}\hfill
    \end{figure}
\end{titlepage}

\begin{enumerate}
\item{
    \textbf{Preguntas de repaso:}
    \begin{itemize}
    \item{¿Qué es una dependencia funcional y cómo se define?\\
        Una dependencia funcional es una relación unidireccional entre dos
        atributos tal que para algún valor de B, solamente tiene relacionado uno
        de A por medio de la relación.\\
        Se definen cómo $A\rightarrow B$.
    }
    \item{¿Para qué sirve el concepto de dependencia en la normalización?\\
        Para eliminar o en su defecto reducir la redundancia en un base de datos.
    }
            funcionales que implica A.\\
    \item{Sea A la llave de R(A, B, C). Indica todas las dependencias
        $$A\rightarrow B;\ \ A\rightarrow C$$
        Y, cómo relación trivial, $A\rightarrow A$.
    }
    \item{¿Qué es una forma normal? ¿Cuál es el objetivo de normalizar un modelo
            de datos?\\
        Una forma normal es una manera de, por medio de reglas sobre las
        relaciones, descomponer los datos de la base de tal manera que se haya
        reducido al redundancia.\\
        El objetivo de normalizr es un modelo de datos es reducir en la mayor
        medida posible la redundancia en una base de datos.
    }
    \item{¿En qué casos es preferible lograr 3NF en vez de BCNF?\\
        Cuándo no se desea descomponer más el esquema.
    }
    \end{itemize}
}
\item {Proporciona algunos ejemplos que demuestren que las siguientes reglas
	no son válidas:
\begin{enumerate}
	\item Si A → B, entonces B → A\\
	Sea la relación R(restaurante,ciudad,teléfono).\\
	Se tiene la DF $restaurante \rightarrow ciudad$, ya que un restaurante tiene
	asociada una sola ciudad, pero notemos que la DF $ciudad \rightarrow
	restaurante$ no se da, ya que una ciudad tiene varios restaurantes. \\
	Por lo que A $\rightarrow$ B no implica que B $\rightarrow$ A
	
	\item Si AB → C, entonces A → C y B → C
	
	Sea la relación R(asignatura,alumno,grupo).
	
	Se tiene que la DF $asignatura-alumno \rightarrow grupo$, ya que cada alumno acude a una asignatura , y esa asignatura con ese alumno tiene un grupo asignado especifico, pero notemos que la DF $asignatura \rightarrow grupo$ y $alumno \rightarrow grupo$ no se dan ya que , dicha asignatura puede ser dada a distintos grupos y un alumno puede estar inscrito en varios grupos.  
\end{enumerate}}
\item Para cada uno de los esquemas que se muestran a continuación:
\begin{enumerate}
	\item R(A,B,C,D,E) con F = \{AB $\rightarrow$ CD, E $\rightarrow$ C, D
		$\rightarrow$ B\}
	\item R(A,B,C,D,E)con F = \{AB $\rightarrow$ C, DE $\rightarrow$ C, B $\rightarrow$ D\}
\end{enumerate}
\begin{itemize}
	\item Especifica de  ser  posible dos DF  no  triviales que  se  pueden
	derivar  de  las  dependencias funcionales dadas.\\
	a. \{AB $\rightarrow$ C, AB $\rightarrow$ D\}.\\
	b. \{BE $\rightarrow$ C, AB $\rightarrow$ AD , AB $\rightarrow$ CAD\}.
	
	\item Indica alguna llave candidata para R.\\
	a.\{ABE\}+=\{ABCDE\}\\
	b.\{ABE\}+=\{ABCDE\}
	
	\item Especifica todas las violaciones a la BCNF\\
	a.\{AB $\rightarrow$ CD, E $\rightarrow$ C, D
	$\rightarrow$ B\}
	
	b.\{AB $\rightarrow$ C, DE $\rightarrow$ C, B $\rightarrow$ D\} 
	\item Normaliza de acuerdo a BCNF, asegúrate de indicar cuáles son las
	relaciones resultantes con sus respectivas dependencias funcionales:\\
	a. Tomamos la violación AB $\rightarrow$ CD.\\
	Obtenemos las relación S(A,B,C,D) con dependecias \{AB $\rightarrow$ CD,
	D $\rightarrow$ B\} y la relación T(A,B,E) con dependencias \{ABE $\rightarrow$ ABE\}.\\
	\{AB\}+=\{ABCD\} es una llave para S, entonces tomamos la violación D $\rightarrow$ B.\\
	Obtenemos la relación U(D,B) con dependencia \{D $\rightarrow$ B\} y la
	relación V(D,A,C) con dependencia \{DAC $\rightarrow$ DAC\}.\\
	Por lo tanto el esquema en BCNF es U(D,B), V(D,A,C) y T(A,B,E).
	
	b. Tomamos la violación B $\rightarrow$ D.
	Obtenemos la relación S(B,D) con dependencias \{B $\rightarrow$ D\} y la relación T(B,A,C,E) con dependencias \{AB $\rightarrow$ C \} y perdemos la dependecia \{DE $\rightarrow$ C\}.
	
La relación S ya esta en forma BCNF mientras que  T no, asi que encontramos una llave para T que sera ABE y dividimos T en las relaciones U(A,B,C) con dependencia \{AB $\rightarrow$ C\} y la relación V(A,B,E) sin ninguna dependencia.

Asi concluyendo que la relación U y V ya estan en BCNF y terminanando la Normalizacion




	
\end{itemize}
\item Para cada una de  las  siguientes  relaciones  con  su  respectivo  conjunto  de  dependencias funcionales:
\begin{enumerate}
	\item R(A,B,C,D,E,F)con F = \{B $\rightarrow$ D, B $\rightarrow$ E, D $\rightarrow$ F, AB $\rightarrow$ C\}
	\item R(A,B,C,D,E)con F = \{A $\rightarrow$ BC, B $\rightarrow$ D, CD$\rightarrow$ E, E $\rightarrow$ A \}
\end{enumerate}
\begin{itemize}
	\item Indica todas las violaciones a la 3NF\\
	b.\{A\}+ = \{ABCDE\} y \{E\}+ = \{EABCD\} son llaves, entonces la única
	violación a la 3NF es B $\rightarrow$ D.
	\item Normaliza de acuerdo a la 3NF\\
	b. Superfluos por la izquiera:  CD $\rightarrow$ E\\
	¿C es superfluo? D $\rightarrow$ E, \{D\}+=\{D\} entonces C no es superfluo.\\
	¿D es superfluo? C $\rightarrow$ E, \{C\}+=\{C\} entonces D no es superfluo.\\
	Superfluos por la derecha: A $\rightarrow$ BC\\
	¿B es superfluo? A $\rightarrow$ C, F'=\{A$\rightarrow$C, B$\rightarrow$D, CD$\rightarrow$E, E$\rightarrow$A\}.\\
	\{A\}+= \{AC\} por lo tanto, B no es superfluo.\\
	¿C es superfluo? A$\rightarrow$B F''=\{A$\rightarrow$B, B$\rightarrow$D, CD$\rightarrow$E, E$\rightarrow$A\}.\\
	\{A\}+= \{AB\} por lo tanto, C no es superfluo.\\
	Entonces F ya es el mínimo.\\
	S(A,B,C), T(B,D), U(C,D,E) y V(E,A) es el esquema en 3NF.
\end{itemize}
\item Sea el esquema: 
$$\textbf{R(A,B,C,D,E,F) con F={BD → E, CD → A, E → C, B → D}}$$

\begin{enumerate}
	\item ¿Qué puedes decir de \textbf{\{A\}+} y \textbf{\{F\}+}? 
	
Sus Cerraduras solo se contienen a ellas mismas , por lo que no pueden alcanzar ningun otro atributo (\{A\}+ = \{A\} y \{F\}+ = \{F\}).

	\item Calcula \textbf{{B}+}, ¿qué puedes decir de esta cerradura? 
	
	La cerradura es \{B\}+ = \{BDECA\} y alcanza a casi todos los atributos de R excepto de F por lo cual no es una llave. 
	
	\item Obtén todas las \textbf{llaves candidatas}. 
	
	La unica que cumple con la definición de llave candidata seria \textbf{BF} por el análisis que hicimos en el inciso anterior
	
	\item ¿R cumple con \textbf{BCNF}? ¿Cumple con \textbf{3NF}? (en caso contrario normaliza)
	
	Empezaremos con BCNF, podemos observar que todas las dependencias son violaciones por lo tanto eligiremos la violacion:  \textbf{CD $\rightarrow$ A};
	por lo tanto obtebdriamos las relaciones S(C,D,A) con la unica dependencia CD $\rightarrow$ A y T(C,D,B,E,F) con las dependencias BD $\rightarrow$ E, E $\rightarrow$ C y B$\rightarrow$ C.
	
	Ya que la relación S esta en BCNF , pero en T siguen presentandose violaciones ya que ninguna tiene una superllave, asi que seguimos con el proceso ahora tomandonos:
	
	\textbf{E$\rightarrow$ C};
	por lo tanto obtendriamos las relaciones U(E,C) con la unica dependencia E $\rightarrow	$ C y V(E,D,B,F) con dependencias BD $\rightarrow$ E y B $\rightarrow$ D .
	
	Ya que la relación U ya esta en BCNF, pero V no tiene del lado izquierdo una superllave , asi que sus dos dependencias faltantes siguen siendo violaciones, asi que seguimos con el procso ahora tomandonos:
	
	\textbf{BD $\rightarrow$ E};	
	por lo tanto obtendriamos las relaciones W(B,D,E) con dependencias BD $\rightarrow$ E y B$\rightarrow$ D y X(B,D,F) con la unica dependencia B$\rightarrow$ D.
	
	Ya que la relación W ya esta en BCNF, pero en X tiene otra superllave que es BF lo cual nos lleva a que su dependencia sigue siendo una violación, por lo que continuamos con el proceso y nos tomamos esta ultima violación:
	
	\textbf{B $\rightarrow$ D};
	por lo que al final tendriamos las relaciones Y(B,D) y Z(B,F) que ambas ya estan en BCNF terminando la Normalización.
	\\\\
	Ahora pasaremos a la 3NF; para hacer la normalización, iniciemos con
  la búsqueda de atributos superfluos. Consideremos la relación
  BD $\rightarrow$ E.
  \begin{itemize}
    \item És D superfluo?  Queremos ver si podemos sustituir
      BD $\rightarrow$ E por B $\rightarrow$ E en F.   Calculemos \{B\}+ =
      \{BDECA\}. Como E aparece en la cerradura de B,
      concluimos que D sí es superfluo.
  \end{itemize}
  De este modo, el nuevo conjunto de dependencias funcionales
  es \{B $\rightarrow$ E, CD $\rightarrow$ A, E $\rightarrow$ C, B $\rightarrow$ D\}. Podemos obtener el conjunto equivalente
  $\{B \to ED, CD \to A, E \to C\}$.  Continuando con la búsqueda
  de atributos superfluos por la izquierda, consideremos a la
  dependencia funcional CD $\rightarrow$ A.
  \begin{itemize}
    \item És C superfluo? Para poder sustituir CD $\rightarrow$ A por
      D $\rightarrow$ A, necesitamos que A aparezca en la cerradura de
      D.   Sin embargo, \{D\}+ = \{D\}, por lo que C no
      es superfluo.

    \item És D superfluo? Para poder sustituir CD $\rightarrow$ A por
      C $\rightarrow$ A, necesitamos que A aparezca en la cerradura de
      C.   Sin embargo, \{C\}+ = \{C\}, por lo que D no
      es superfluo.
  \end{itemize}
  Ni C ni D son superfluos.   Procedemos a buscar atributos
  superfluos por la derecha.   Consideremos la dependencia funcional
  B $\rightarrow$ ED.
  \begin{itemize}
    \item És E superfluo?   Consideremos a $F' = \{ B \to D,
      CD \to A, E \to C\}$, y calculemos $\{B\}^+$ con este nuevo
      conjunto $\{B\}^+ = \{BD\}$.   Como $E$ no aparece en la
      cerradura, concluimos que no es superfluo.

    \item És D superfluo?   Consideremos a F' = \{B $\rightarrow$ E,
      CD $\rightarrow$ A, E $\rightarrow$ C\},y calculemos \{B\}+ con este nuevo
      conjunto \{B\}+ = \{BEC\}. Como D no aparece en la
      cerradura, concluimos que no es superfluo.
  \end{itemize}
  Así, nuestro conjunto de dependencias funcionales mínimo
  es 
  
    $F_{\{MIN\}}$ = \{B $\rightarrow$ ED, CD $\rightarrow$ A, E $\rightarrow$ C\}.
  
  Por lo que hacemos una nueva relación para cada dependencia
  funcional, y como en ninguna de ellas aparece una llave candidata
   añadimos una relación
  adicional con los atributos de la llave y la dependencia funcional
  trivial:
  \begin{itemize}
    \item $S(B, E, D)$ con la dependencia funcional $B \to ED$.

    \item $T(C, D, A)$ con la dependencia funcional $CD \to A$.

    \item $U(E, C)$ con la dependencia funcional $E \to C$.

    \item $V(B, F)$ con la dependencia funcional $BF \to BF$.
  \end{itemize}
	\item Se ha decidido dividir \textbf{R} en las siguientes relaciones \textbf{S(A,B,C,D,F)} y \textbf{T(C,E)}, ¿se puede recuperar la información de \textbf{R}? 
	
	 No, porque se pierderia la dependencia \textbf{BD $\rightarrow$ E}.

\end{enumerate}
\item Para  cada  uno  de  los  esquemas,  con  su  respectivo  conjunto  de  dependencias  multivaluadas, resuelve los siguientes puntos:\\
\begin{enumerate}
	\item R(A,B,C,D)con DMV = {AB $\twoheadrightarrow$C, B $\rightarrow$ D}
	\item R(A,B,C,D,E)con DMV = { A $\twoheadrightarrow$B, AB $\rightarrow$ C, A $\rightarrow$ D, AB $\rightarrow$ E}\\
\end{enumerate}
\begin{itemize}
	\item Encuentra todas las violaciones a la 4NF.
	\{AB\}+=\{ABCDE\} es una llave. Las violaciones son \{A$\twoheadrightarrow$B\}.\\
	\item Normaliza de acuerdo a la 4NF.\\
	Tomamos la violación A$\twoheadrightarrow$B.\\
	Obtenemos las relaciones S(A,B) con DMV=\{A$\twoheadrightarrow$B\} y
	T(A,C,D,E) con DMV=\{A$\rightarrow$\}.\\
	\{ACE\}+=\{ACED\} es una llave para T, entonces tomamos la violación
	A$\rightarrow$D.\\
	Obtenemos las relaciones U(A,D) con DMV=\{A$\rightarrow$\} y V(A,C,E) con
	DMV=\{ACE$\rightarrow$\}.\\
	Por lo tanto, el esquema en 4NF es S(A,B), U(A,D) y V(A,C,E).
	
\end{itemize}

	\item Se tiene la siguiente relación: 
	$$\textbf{R(idEnfermo, idCirujano, fechaCirugía, nombreEnfermo, direcciónEnfermo, nombreCirujano, nombreCirugía, medicinaSuministrada, efectosSecundarios)}$$
	\begin{itemize}
		\item Expresa las siguientes restricciones en forma de \textbf{dependencias funcionales}: 
		\begin{enumerate}
		\item A un enfermo sólo se le da una medicina después de la operación.
		
		\textbf{R:} $idEnfermo,fechaCirugia,nombreCirugia \rightarrow medicinaSuministrada$

		\item Si existen efectos secundarios estos dependen sólo de la medicina suministrada. Sólo puede existir un efecto secundario por medicamento. 
		
		\textbf{R:} $medicinaSuministrada \rightarrow efectosSecundarios$
		\end{enumerate}
	\end{itemize}
\end{enumerate}

\end{document}
