\documentclass[12pt]{article}
\usepackage[paper=a4paper,margin=1in]{geometry}
\usepackage[utf8]{inputenc}
\usepackage{amsmath}
\usepackage{amssymb}
\usepackage{enumerate}
\usepackage{graphicx}
\graphicspath{img/ }
\usepackage{booktabs}
\usepackage{lipsum}
\graphicspath{{img/}}
\usepackage{multirow}
\usepackage[dvipsnames]{xcolor}
\input{kvmacros}

\begin {document}
\begin{enumerate}
	\item ¿Por qué elegirías almacenar datos en un sistema de base
	 de datos en lugar de simplemente almacenarlos utilizando el
	 sistema de archivos de un sistema operativo? ¿En qué casos no
	 tendría sentido utilizar un sistema de base de datos?\\

	 No podría guardar cantidades grandes de información, el acceso
	 y las busquedas de información grandes	toman demasiado tiempo
	 en un sistema de archivos.Tampoco se pueden ordenar y actualizar
	 constantemente, tampoco podría modificar el esquema de la base
	 de datos sin tener que modificar completamente los archivos.\\
	 El sistema de archivos no es seguro incluso con un sistema
	 de contraseñas ya que no podemos asegurar la integridad de los
	 datos. No es dinámico, para que más personas tuvieran acceso a
	 mi base de datos tendría que pasarles los datos manualmente, lo cual es muy ineficiente.\\
	 No tendría sentido usar una base de datos cuando se tienen muy
	 pocos datos a guardar.

	\item ¿Qué ventajas y desventajas encuentras al trabajar con una
	 base de datos?\\
	 \textbf{Ventajas:}
	 \begin{itemize}
	 	\item Se pueden sincronizar datos en ella
	 	\item Se puede modificar la estructura fácilmente
	 	\item Garantizan la fiabilidad
	 	\item Existen por un periodo largo de tiempo
	 	\item Permite controlar la redundancia
	 	\item Son independientes a los programas que proporcionan las
	 	vistas.
	 \end{itemize}
 	 \textbf{Desventajas:}
 	 \begin{itemize}
 	 	\item Son costosas.
 	 	\item No cualquiera puede manejarla, se necesita alguien
 	 	especializado.
 	 	\item Se requiere de capacitación para su manejo.
 	 \end{itemize}
	 \item Investiga  cuáles  serían  los distintos tipos de
	  usuarios finales de una base de datos, indica las principales
	  actividades que realizaría cada uno de ellos. \\\\
	  \textbf{Usuarios Finales.} Acceden a la base de datos desde
	  alguna terminal, pueden utilizar un lenguaje de consultar o un
	  programa de aplicación. Tenemos distintos tipos de usuarios
	  finales:
	  \begin{itemize}
	  	\item \textbf{Esporádicos.} Acceden de vez en cuando, no
	  	siempre requieren la misma información. Utilizan lenguajes
	  	sofisticados de consulta para especificar su solicitud.\\
	  	\item \textbf{Paramétricos.} Estos usuarios hacen consultas
	  	y actualizan la base de datos constantemente, no necesitan
	  	aprender del lenguaje de consultas, ya que normalmente
	  	utilizan una interfaz gráfica diseñada para ese propósito.
	  	\item \textbf{Sofisticados o avanzados.} Son profesionales,
	  	tales como ingenieros, científicos y otros que están muy
	  	familiarizados con el sistemas manejador de bases de datos.
	  	Estos usuarios suelen hacer uso de sus conocimientos para
	  	satisfacer requerimientos complejos.
	  	\item \textbf{Autónomos.} Mantienen sus propias bases de datos, utilizando paquetes de programas que facilitan su uso.

	  \end{itemize}

      \item Indica las principales características de los modelos de
      datos más representativos. ¿Cuáles serían las diferencias
      entre los  modelos relacional, orientado a objetos,
      semiestructurado y objeto –relacional?\\
      \textbf{Modelo Relacional.}Los datos se perciben como tablas,
      es un sistema cerrado, todas las operaciones son siempre tablas.\\
      \textbf{Modelo Orientado a Objetos.} Los datos se modelan como
      objetos, se tiene comportamiento (métodos o funciones) y estado.\\
      \textbf{Modelo Semiestructurado.} Es una colección de nodos y
      cada nodo tiene datos con diferentes esquemas, esto lo hace un
      sistema menos rígido.\\
      \textbf{Modelo Objeto-Relacional} Representa los datos como tablas,
      permite construir tipos de objetos complejos, capacidad para
      encapsular y asociar métodos a los objetos.

      \item Supón  que  un banco pequeño desea  almacenar  su
      información  en una  base  de  datos  y  le gustaría comprar
      el SMBD que  tenga  la  menor  cantidad  de  características
      posibles. Está interesado en ejecutar la aplicación en una sola computadora personal y no se planea compartir la
      información  con  nadie.  Para  cada  una  de  las
      siguientes  características  explica  por  qué  se debería o
      no incluir en el SMBD que se desea comprar (suponiendo que se
      pueden comprar por separado): \textbf{seguridad, control de
      concurrencia, recuperación en caso de fallas, lenguaje de
      consulta, mecanismo de vistas, manejo de transacciones}\\
      \textbf{Seguridad:} Es necesario comprarlo, ya que no quieren
      compartir la información con nadie, por lo tanto los datos
      se deben proteger y evitar que cualquiera pueda acceder a
      ellos.\\
      \textbf{Control de concurrencia:} No se necesita incluir ya que
      se planea ejecutar la aplicación en una sola computadora
      personal, por lo que el control de concurrencia sería
      completamente innecesario si no va a ser utilizados por varias
      personas a la vez.\\
      \textbf{Recuperación en caso de fallas:} Es necesario, ya que
      la aplicación solo funcionará en una sola computadora personal,
      por lo que si hubiera algún tipo de falla, toda la información
      se perdería y no habría ningún otro lugar para recuperarlo.\\
      \textbf{Lenguaje de consulta:} Es necesario, para poder acceder,
      actualizar y guardar los datos en la base de datos.\\
      \textbf{Mecanismo de vista} No es necesario, si solo va a ser
      utilizado por una persona...
      \textbf{Manejo de transacciones}No es necesario\\

      \item ¿Qué es la Calidad de Datos y cómo se relaciona con las
      bases de datos?\\
      Es  el estado de completez, validez, consistencia y exactitud
      que hace a los datos apropiados para su uso. También puede ser
      definida como el grado en el que un conjunto de características
      cumplen con los requerimientos necesarios.
      Se relaciona con las bases de datos ya que una buena calidad de
      datos es esencial en la resolución de entidades, ya que aumenta la fiabilidad de las entidades resueltas y las relaciones
      detectadas.Muchos problemas que se suelen presentar en una base
      de datos se pueden solucionar con una buena calidad y manejo
      de datos. Además esto reduce costos a la larga.



 \end{enumerate}
\begin{thebibliography}
\bibitem{DBA}{https://pt.wikipedia.org/wiki/Administrador_de_banco_de_dados}
\end{thebibliography}
\end {document}
