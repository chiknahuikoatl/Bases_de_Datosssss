\documentclass[a4paper, 12pt]{report}

\usepackage[spanish]{babel}
\usepackage[utf8]{inputenc}
\usepackage[left=4cm, right=4cm, top=4cm]{geometry}
\usepackage{textcomp}
\usepackage{booktabs}
\usepackage{amssymb}
\usepackage{bussproofs}
\usepackage{fancyhdr}
\usepackage{graphicx}
\usepackage{amsmath}

\usepackage{hyperref}
\hypersetup{
    colorlinks=true,
    linkcolor=blue,
    filecolor=magenta,
    urlcolor=cyan,
}

\pagestyle{fancy}
\lhead{Almeida, Figueroa \& Ibarra}
\chead{Tarea 2}
\rhead{\today}

\begin{document}
\begin{titlepage}
    \centering
    {\scshape\Huge Universidad Nacional Autónoma de México \par}
    \vspace{1.25cm}
    {\scshape\huge Fundamentos de Bases de Datos\par}
    \vspace{1.25cm}
    {\huge\bfseries Tarea 2: Modelo Entidad-Relación\par}
    \vspace{1.25cm}
    {\Large\textsc Almeida Rodríguez Jerónimo\par}
    \vspace{.1cm}
    {\large\texttt{418003815}\par}
    \vspace{0.25cm}
    {\Large\textsc Figueroa Sandoval Gerardo Emiliano\par}
    \vspace{.1cm}
    {\large\texttt{315241774}\par}
    \vspace{0.25cm}
    {\Large\textsc Ibarra Moreno Gisselle \par}
    \vspace{.1cm}
    {\large\texttt{315602193}\par}
    \vspace{1.5cm}
    \vfill
    \begin{figure}[hb!]
        \includegraphics[width=.3\textwidth]
            {../logos/escudo_f-ciencias.png}\hfill
        \includegraphics[width=.3\textwidth]
            {../logos/Escudo_UNAM.png}\hfill
    \end{figure}
\end{titlepage}
\begin{enumerate}
\item[1)]{
\begin{enumerate}
    \item[i)]{Un conjunto de entidades débiles siempre se puede convertir en un
        conjunto de entidades fuertes añadiéndole a sus atributos la llave
        primaria del conjunto de entidades fuertes a las que está asociado.
        Describe qué tipo de redundancia resultaría si se realizara dicha
        conversión.\\
        La redundancia que resultaría sería que cualquier dato que se haya
        tomado cómo llave primaria en la entidad principal estaría repetida
        tantas veces cómo entidades débiles asociadas a dicha entidad.
    }
    \item[ii)]{}
    \item[iii)]{}
    \item[iv)]{}
\end{enumerate}
}
\item[2)]{
\begin{enumerate}
    \item[a)]{}
    \item[b)]{}
    \item[c)]{}
\end{enumerate}
}
\end{enumerate}


\begin{thebibliography}{20}
\end{thebibliography}


\end{document}
