\documentclass[a4paper, 12pt]{report}

\usepackage[spanish]{babel}
\usepackage[utf8]{inputenc}
\usepackage[left=4cm, right=4cm, top=4cm]{geometry}
\usepackage{textcomp}
\usepackage{booktabs}
\usepackage{amssymb}
\usepackage{bussproofs}
\usepackage{fancyhdr}
\usepackage{graphicx}
\usepackage{amsmath}

\usepackage{hyperref}
\hypersetup{
    colorlinks=true,
    linkcolor=blue,
    filecolor=magenta,
    urlcolor=cyan,
}

\pagestyle{fancy}
\lhead{Almeida, Figueroa \& Ibarra}
\chead{Tarea 3}
\rhead{\today}

\begin{document}
\begin{titlepage}
    \centering
    {\scshape\Huge Universidad Nacional Autónoma de México \par}
    \vspace{1.25cm}
    {\scshape\huge Fundamentos de Bases de Datos\par}
    \vspace{1.25cm}
    {\huge\bfseries Tarea 3: Modelo Relacional\par}
    \vspace{1.25cm}
    {\Large\textsc Almeida Rodríguez Jerónimo\par}
    \vspace{.1cm}
    {\large\texttt{418003815}\par}
    \vspace{0.25cm}
    {\Large\textsc Figueroa Sandoval Gerardo Emiliano\par}
    \vspace{.1cm}
    {\large\texttt{315241774}\par}
    \vspace{0.25cm}
    {\Large\textsc Ibarra Moreno Gisselle \par}
    \vspace{.1cm}
    {\large\texttt{315602193}\par}
    \vspace{1.5cm}
    \vfill
    \begin{figure}[hb!]
        \includegraphics[width=.3\textwidth]
            {../logos/escudo_f-ciencias.png}\hfill
        \includegraphics[width=.3\textwidth]
            {../logos/Escudo_UNAM.png}\hfill
    \end{figure}
\end{titlepage}

\section*{1. Preguntas de Repaso.}{
}

\section*{2. Modelo Relacional.}{
}

\section*{3. Lectura.}{
\begin{enumerate}
\item[1.]{\textbf{Regla de Información: } \\
    \textit{Toda la Información en la base de datos está representada de una
    manera única, cómo valores en tablas.}\\
    En general, toda la información se guarda en tablas.
}
\item[2.]{\textbf{Regla del Acceso Garantizado: }\\
    \textit{Está garantizado que cada dato (valor atómico) pueda ser accedido
    lógicamente por medio de una combinación de nombre de tabla, valor de la
    llave primaria y el nombre de la columna.}\\
    Esta regla lo que busca es garantizar el acceso a la información de manera
    única por medio de un conjunto de ``coordenadas'' que, basadas en el nombre
    de la tabla, la llave primaria y el nombre de la columna, permite que cada
    dato pueda recuperarse de manera única.
}
\item[3.]{\textbf{Tratamiento Sistematico de los Valores 'NULL': }\\\textit{
    Los valores NULL son usados en SMBDR para representar informanción faltante
    o desconocida de una manera sistemática, independientemente del tipo de dato.
    Son distintos del caracter de cadena vacía, de caracteres blancos y de
    cualquier número.}\\
    Implementa el uso de valores NULL, que en general significan que el valor
    es desconocido. Usualmente se trata cómo operar el vacío ($\O$) en teoría de
    conjuntos: cualquier cosa operada con NULL devuelve NULL; aunque en algunos
    casos, si se intenta concatenar con una cadena, devuelve la nueva cadena.
}
\item[4.]{\textbf{Catálogo Dinámico en Línea Basado en el Modelo Relacional:}\\
    \textit{En el nivel lógico, la descripción de la base de datos está
    representada al mismo nivel que los datos ordinarios. De esta manera, los
    usuarios autorizados pueden usar el mismo lenguaje relacional que se usa
    para datos regulares.}\\
}
\item[5.]{\textbf{Regla del Sublenguaje de Datos Comprensivos:}\\
    \textit{
    Un sistema relacional puede soprtar varios lenguajes de programación y modos
    de uso terminal, pero al menos uno de ellos debe tener expresiones que sean
    expresables por medio de una sitáxis bien definida y puede soportar lo
    siguiente:}\\
    \begin{itemize}
        \item{Definición de datos.}
        \item{Visibilización de la definición.}
        \item{Manipulación de datos}
        \item{Restricciones de integridad.}
        \item{Autorizaciones.}
        \item{Bordes de transacciones (inicio, commit, etc.).}
    \end{itemize}
}
\item[6.]{\textbf{Regla de la vista actualizada:}\\\textit{
    Todas las vistas que se pueden actualizar las puede actualizar el sistema.}
}
\item[7.]{\textbf{Insert, Delete y Actualizar a Alto Nivel:}\\\textit{
    La abilidad de manejar una relación base o derivada cómo un único operando
    no solo al recuperar la información sino también para insertar, borrar y
    actualizar información.}\\
}
\item[8.]{\textbf{Independencia Física de Datos:}\\\textit{
    Los programas de aplicación y actividades terminales se mantienen funcinando
    cuándo hay cambios en la representación de almacenamiento o métodos de
    acceso.}\\
}
\item[9.]{\textbf{Independencia Lógica de Datos:}\\\textit{
    Los programas de aplicación y actividades terminales se mantienen lógicamente
    funcionales cambios que preserven la información se haga en alguna tabla.}\\
}
\item[10.]{\textbf{Independencia de Integridad:}\\\textit{
    Las restricciones de integridad específicas de de una base relacional deben
    poder ser definidas en el lenguaje de el lenguaje relacional y deben poder
    ser almacenadas en el catálogo, no en los programas de aplicación. Protege la
    información de datos inválidos.}\\
}
\item[11.]{\textbf{Independencia de Distribución:}\\\textit{
    El lenguaje del SMBD debe permitir que los programas de aplicación y actividades
    terminales permanezcan lógicamente funcionales y accesibles ya sea que los datos
    estén físicamente cnetralizados o distribuidos.}\\
}
\item[12.]{\textbf{Regla de No-Subversión:}\\\textit{
    Si u sistema relacional permite el uso de slgún lengiaje de programación de
    un nivel más bajo, este lenguaje no puede ser usado para traspasar las reglas
    de integridad y restricciones establecidas por el lenguaje relacional.}\\
}
\end{enumerate}
}




\end{document}
